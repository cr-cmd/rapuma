% This is an example lccode.tex file and it used in the hyphenation
% process. This needs to have a good explanation written as to what
% this file is for and how to manage it, but alas we've run out of
% time. Just know that for hyhpenation to work properly, this file
% must be present in a group where hyphenation is expected to work.

\lccode "2011 = "2011	% Allow TeX hyphenation to ignore a Non-break hyphen
\lccode "0029 = "0029
\lccode "0021 = "0021
\lccode "002E = "002E
\lccode "002C = "002C
\lccode "003A = "003A
\lccode "003F = "003F
\lccode "003B = "003B
\lccode "2019 = "2019
\lccode "201D = "201D
\lccode "200C = "200C
\lccode "200D = "200D
\lccode "2026 = "2026
\catcode "2011 = 11	% Changing the catcode here allows the \lccode above to work
