% File: hyphenation.tex

% The data in this file is for the purpose of driving the hyphenation
% process in this project. Comments are made below to guide in the
% adding or changing data. Please refer to the documentation for more
% details on how to make changes or how the process works.

% The following are hyphenation settings that will be used for this
% publishing project. More settings can be added here as well. For more
% information about what settings may enhance hyphenation for this
% project, look in "The TeXbook", by Donald E. Knuth where you will
% find much more on the subject.

% Define a language
\newlanguage\<ISO Code>language

% Set the language for this project to the one you just defined
\language = \<ISO Code>language

% Define the hyphenation character you will use (default = "002D)
\defaulthyphenchar="002D

% TeX \hyphenpenalty setting to 10000 which effectivly turns off
% hyphenation.
\hyphenpenalty=0

% Like with \hyphenpenalty this will adjust TEX's ability to break
% explicit hyphens.  These are words that occure in the text which
% are spelled with a hyphen in them.  Normally, TeX will not touch
% these unless this is set to a low number like -50. This setting
% is active when useHyphenation is set to True. Default = 50
\exhyphenpenalty=50

% Pretolerance is an integer parameter that is used in TeX's line
% breaking algorithm.  Use an integer from -1 to 10000.  If \pretolerance
% is -1, then TeX bypasses the first attempt at breaking a paragraph
% without hyphenation.  Normally it is always best to allow TeX to use
% its defaults first so the setting here is 0 (zero) by default.
\pretolerance=0

