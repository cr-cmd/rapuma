% extensions.tex
%
% Custom TeX setup file for the usfmTex macro package.

%%%%%%%%%%%%%%%%%%%%%%%  Extra Tweaks %%%%%%%%%%%%%%%%%%%%%%

% In a perfect world none of these would be needed but when
% your publication throws you a curve, perhaps one of these
% work-arounds might help you do what you want to do.

% Process \b 
% This is often frowned upon but if you want to add extra
% spaces around poetry, uncomment this next line.
%\def\b{\vskip 0.5\baselineskip}

%%%%% Baselineskip Adjustment Hook
% This hook provides a means to adjust the baselineskip on a
% specific style. It provides a place to put the initial 
% setting so the hook can make the change and then go back
% to the initial setting when done.
\newdimen\remblskip \remblskip=\baselineskip

% Baselineskip Adjustment Hook Example
%\sethook{start}{s1}{\remblskip=\baselineskip \baselineskip=10pt}
%\sethook{after}{s1}{\baselineskip=\remblskip}

%%%%% Substituting Characters
% Some times, when a character does not exist in a font
% you can substitute from another if no special rendering
% is needed. This code will do that. Modify as needed.
%% Example (1)
%\font\cwi="[../Fonts/Padauk/Padauk.ttf]" at 10pt
%\catcode"A92E=\active                            % Make U+A92E an active character
%\def^^^^a92e{\leavevmode{\cwi\char"A92E}}        % Define it to print itself

%% Example (2)
%\font\crossmaltese="[../Fonts/freefont/FreeSerif.ttf]" at 12pt
%\catcode"2720=\active                            % Make U+F058 an active character
%\def^^^^2720{\leavevmode{\crossmaltese\char"2720}}        % Define it to print itself

%%%%% Non-standard Spaces
% Some publications may use non-standard (U+0020) between words.
% But TeX (and XeTeX) will treat spaces other than U+0020 as
% non-breaking which messes up your justification. This is a
% work around to force TeX to break and stretch words with
% another space character in a controled way.
%\catcode"2009=13
%\def^^^^2009{\hskip .2em plus.1em minus.1em\relax}

%%%%% Heading space
% There always seems to be problems with extra space between 
% the section heading and the top of the column when the 
% section head is at the top of the column. To take up the
% slack this code will usually help. Any adjustments needed
% should be done in the .sty and \VerticalSpaceFactor.
% Trying to adjust this code doesn't seem to make any dif.
%\catcode`\@=11
%\def\gridb@x#1{%
 %\setbox0=\ifgridp@c\vbox{\box#1}\else\killd@scenders#1\fi%
 %\dimen2=\ht0 \advance\dimen2 by \dp0
 %\dimen0=\baselineskip \vskip\baselineskip
 %\ifgridp@c\line{}\nobreak\fi % otherwise first \line in loop won't get any baselineskip
                              % when doing a picture box, because it's not part of the
                              % current page
% \loop \ifdim\dimen0<\dimen2
%   \advance\dimen0 by \baselineskip
%    \line{}\nobreak \repeat
% \setbox0=\vbox to 0pt{\kern-\ht0\unvbox0}
% \unvbox0 \nobreak
%}
%\catcode`\@=12

%%%%%%%%%%%%%%%%%%%%% TOC Generation %%%%%%%%%%%%%%%%%%%%%%

% The following code controls the layout of the TOC file
% This code can be addjusted to get the right look to your
% TOC. You can also adjust the \toc<n> styles found in
% the .sty file. The functions that are embedded in the
% TOC file are:
%	\tbltwowlheader{lable1}{label3}
%	\tbltwowlrow{val1}{val3}
% Or, if you wish a three column layout you could use this:
%	\tblthreewlheader{lable1}{label3}
%	\tblthreewlrow{val1}{val3}
% Other layouts, more code will be need to be added.
% To change the leader, leader spacing and other general
% layout formating, adjust the settings below.

\catcode`@=11

\def\makedigitsother{\m@kedigitsother}
\def\makedigitsletters{\m@kedigitsletters}

% These are custom settings that can go in the .tex file to
% adjust elements on the page
\def\myleader{.}		% The leader character
\def\leaderspace{0.6em}		% The space between the leader characters
\def\tblmarginright{0in}	% The width of the table's right margin
\def\tblheaderspace{4pt}	% The space between the header and the first row
\def\tblrowspace{4pt}		% The space between rows of the table
\def\tblthreewcolmngap{.75in}	% The space between col 1 & 2 on a three col table


% Macro code for leadered two colum layout:
\newdimen\tbltwowllabel
\def\tbltwowlheader#1#2{\parfillskip=\tblmarginright\toch #1\toch*\hfil
  \setbox0=\hbox{\toch #2\toch*}\tbltwowllabel=\wd0\box0\par\vskip \tblheaderspace}

\def\tbltwowlrow#1#2{\parskip=\tblrowspace\parfillskip=\tblmarginright\noindent\tocbn #1\tocbn*%
  \quad\leaders\hbox to \leaderspace{\hss\myleader\hss}\hfill\hbox to \tbltwowllabel{%
  \hfil\tocpg #2\tocpg*}\par}%

% Macro code for leadered three colum layout:
\newdimen\tblthreewllabel
\def\tblthreewlheader#1#2#3{\parfillskip=\tblmarginright\setbox0\hbox{\toch #1\toch*\relax}\tblthreewllabel=\wd0
  \noindent\box0\hskip\tblthreewcolmngap\toch #2\toch*\hfil\toch #3\toch*\par\vskip \tblheaderspace}

\def\tblthreewlrow#1#2#3{\parskip=\tblrowspace\parfillskip=\tblmarginright\noindent\hbox to \tblthreewllabel{\tocbn #1\tocbn*\hss\relax}%
  \hskip\tblthreewcolmngap\tocba #2\tocba*
  \leaders\hbox to \leaderspace{\hss\tocbn \myleader\tocbn*\hss}\hfil
  \enskip\tocpg #3\tocpg*\par}%

\catcode `@=12


