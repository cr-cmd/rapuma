% This TeX code provides a way to have marginal verses
% Caution, it is still under heave development


\catcode`\@=11
\let\myoldv=\v
%\newif\multipleverses
\newtoks\myeverypar
\newdimen\mybaseline
{\catcode`\~=12 \lccode`\~=32 \catcode`\*=12 \lowercase{%
\gdef\marginverse #1 {%
    \gdef\v@rse{#1}% 
    \x@\spl@tverses\v@rse--\relax
    \global\setbox2\hbox{\ifc@ncelfirstverse\else\myoldv ~\printv@rse\myoldv *\fi}%
    \egroup\str@t\dimen0=\ht2\advance\dimen0 by \dp\str@tbox
    \myeverypar=\everypar\everypar={}%
    \dimen1=\columnshift\advance\dimen1 by -\AfterVerseSpaceFactor\FontSizeUnit
    \m@rkverse \the\v@rsehooks \gdef\reference{\ch@pter:\v@rse}%
    \vadjust{\getp@ram{baseline}{v}\ifx\p@ram\relax\let\p@ram\baselineskip\fi
        %\kern -\dimen0\setbox1=\vbox to \dimen0{\llap{\unhbox2\kern\AfterVerseSpaceFactor\FontSizeUnit}}\dp1=0pt\box1}\nobreak
        \kern -\dimen0\setbox1=\vbox{\llap{\vbox{\leftskip=0pt plus 1fill %
            \rightskip=\z@skip\hsize=\dimen1 \baselineskip=\p@ram
            \trace{v}{hs=\the\hsize , bl = \the\baselineskip}%
            \noindent \unhbox2 \marginremovehboxes}%
          \kern\AfterVerseSpaceFactor\FontSizeUnit}}%
        \ht1=\dimen0\dp1=0pt\tracingparagraphs=0
        \trace{v}{\reference - tolerance=\the\tolerance}\box1}\nobreak
    \hskip 1sp%
    \everypar=\myeverypar}}}
\gdef\marginremovehboxes{%
  \setbox0=\lastbox 
  \ifhbox0{\removehboxes}\unhbox0\fi}

\def\myv{%
    \x@\global\x@\let\x@\n@xt\csname after-v\endcsname \n@xt
    \leavevmode\c@ncelfirstversefalse
    \ifnum\spacefactor=\n@wchaptersf \kern0sp %
     \ifOmitVerseNumberOne \c@ncelfirstversetrue \fi\fi
    \bgroup\m@kedigitsother\marginverse} % ensure we are in horizontal mode to build paragraph
\def\initmyverse{\let\v=\myv}
\addtoinithooks{\initmyverse}%
\def\printv@rse{\AdornVerseNumber{\v@rsefrom}\ifx\v@rsefrom\v@rseto
%                \else\setbox0\hbox{-} -\kern-\wd0\kern\wd0\penalty\hyphenpenalty \AdornVerseNumber{\v@rseto}\fi}
                % The following seems to look better than the previous. Is there a way to
                % do some math on \wd0 to make it act more proportionally? Hard coding to
                % -1.5pt doesn't seem like a good way to do it.
                \else\setbox0\hbox{-}-\kern-1.5pt\kern\wd0\penalty\hyphenpenalty \AdornVerseNumber{\v@rseto}\fi}

\endinput
