% ptx2pdf-ext.tex
%
% Custom TeX setup file for the ptx2pdf macro package.

%%%%%%%%%%%%%%%%%%%%%%%  Extra Tweaks %%%%%%%%%%%%%%%%%%%%%%

% In a perfect world none of these would be needed but when
% your publication throws you a curve, perhaps one of these
% work-arounds might help you do what you want to do.

% Process \b 
% This is often frowned upon but if you want to add extra
% spaces around poetry, uncomment this next line.
%\def\b{\vskip 0.5\baselineskip}

%%%%% Baselineskip Adjustment Hook
% This hook provides a means to adjust the baselineskip on a
% specific style. It provides a place to put the initial 
% setting so the hook can make the change and then go back
% to the initial setting when done.
\newdimen\remblskip \remblskip=\baselineskip

% Baselineskip Adjustment Hook Example
%\sethook{start}{s1}{\remblskip=\baselineskip \baselineskip=10pt}
%\sethook{after}{s1}{\baselineskip=\remblskip}

%%%%% Header output
% To adjust the size of the page number in the header or footer
% use the following code. Adjust font name and size as necessary.
%\font\mysmallfont="[../Fonts/CharisSIL/CharisSILB.ttf]" at 10pt
%\def\pagenumber{{\mysmallfont \folio}}

% This will output only the book name in the header that is
% found in \h. (This should be added to ptx2pdf.)
\catcode`\@=11
\def\bookname{\x@\extr@ctfirst\p@gefirstmark\relax\@book}
\catcode`\@=12

%%%%% Footnote tweaks
% Footnote caller kerning - To adjust space around the
% footnote caller use the following code Adjust the kern
% amounts as necessary
\let\OriginalGetCaller=\getcaller
\def\getcaller#1#2{%
  \kern0.2em\OriginalGetCaller{#1}{#2}\kern0.4em}

% Inter Note Skip - Adjust the horizontal space between footnotes,
% both paragraphed and non-paragraphed
\catcode`\@=11
  \intern@teskip=10pt
\catcode`\@=12

% Inter-note Penalty - Control the amount of "tension" between
% parts of a footnote to help control line breaking. If you
% use the highest setting, 10000, it will never break. A lower
% setting, like 9999, will lossen it up. Default is 9999.
\def\internotepenalty{9999}

%%%%% Substituting Characters
% Some times, when a character does not exist in a font
% you can substitute from another if no special rendering
% is needed. This code will do that. Modify as needed.
%% Example (1)
%\font\cwi="[../Fonts/Padauk/Padauk.ttf]" at 10pt
%\catcode"A92E=\active                            % Make U+A92E an active character
%\def^^^^a92e{\leavevmode{\cwi\char"A92E}}        % Define it to print itself

%% Example (2)
%\font\crossmaltese="[../Fonts/freefont/FreeSerif.ttf]" at 12pt
%\catcode"2720=\active                            % Make U+F058 an active character
%\def^^^^2720{\leavevmode{\crossmaltese\char"2720}}        % Define it to print itself

%%%%% Non-standard Spaces
% Some publications may use non-standard (U+0020) between words.
% But TeX (and XeTeX) will treat spaces other than U+0020 as
% non-breaking which messes up your justification. This is a
% work around to force TeX to break and stretch words with
% another space character in a controled way.
%\catcode"2009=13
%\def^^^^2009{\hskip .2em plus.1em minus.1em\relax}

%%%%% Heading space
% There always seems to be problems with extra space between 
% the section heading and the top of the column when the 
% section head is at the top of the column. To take up the
% slack this code will usually help. Any adjustments needed
% should be done in the .sty and \VerticalSpaceFactor.
% Trying to adjust this code doesn't seem to make any dif.
% This is normally on by default.
\catcode`\@=11
\def\gridb@x#1{%
 \setbox0=\ifgridp@c\vbox{\box#1}\else\killd@scenders#1\fi%
 \dimen2=\ht0 \advance\dimen2 by \dp0
 \dimen0=\baselineskip \vskip\baselineskip
 \ifgridp@c\line{}\nobreak\fi % otherwise first \line in loop won't get any baselineskip
                              % when doing a picture box, because it's not part of the
                              % current page
 \loop \ifdim\dimen0<\dimen2
   \advance\dimen0 by \baselineskip
    \line{}\nobreak \repeat
 \setbox0=\vbox to 0pt{\kern-\ht0\unvbox0}
 \unvbox0 \nobreak
}
\catcode`\@=12

% Special commands (use with care, may have bad side-effects)
\catcode`@=11
\def\makedigitsother{\m@kedigitsother}
\def\makedigitsletters{\m@kedigitsletters}
\catcode `@=12
\def\nbsp{ }
\def\zwsp{​}
\newdimen\remblskip \remblskip=\baselineskip
\def\suckupline{\vskip -\baselineskip}
\def\suckuphalfline{\vskip -0.5\baselineskip}
\def\suckupqline{\vskip -0.25\baselineskip}
\def\skipline{\vskip\baselineskip}
\def\skiphalfline{\vskip 0.5\baselineskip}
\def\skipqline{\vskip 0.25\baselineskip}


