% File: characters.tex

% The data in this file is for the purpose of driving the hyphenation
% process in this project. Comments are made below to guide in the
% adding or changing data. Please refer to the documentation for more
% details on how to make changes or how the process works.

% For hyphenation to take place, TeX needs to know what characters
% do, and do not, form words. In this file these characters (or letters)
% are defined. The two primary TeX commands used here are \lccode, for
% lower case characters, and \uccode for upper case characters. Brief
% comments are made. For more information on the use of these and other
% character definition commands, please consult "The TeXbook" by Donald
% Knuth, the author of TeX.

% These are standard non-word-forming characters like punctuation and
% other such characters found in the text to be typeset. Edit as needed
% for your writing system. The characters in this list are defined as
% lower case (\lccode) and they are "turned off" with 0 (zero). The code
% used is a Unicode value proceeded by a double quote (").
\lccode "201C = 0
\lccode "201D = 0
\lccode "2018 = 0
\lccode "2019 = 0
\lccode "0028 = 0
\lccode "0029 = 0
\lccode "002C = 0
\lccode "002E = 0
\lccode "0021 = 0
\lccode "003F = 0
\lccode "002A = 0

% In this list we define any non-word forming characters that are
% specifically part of the writing system you are using. The format
% is the same as above.
%\lccode "A92E = 0

% The following are necessary, specify word-forming characters for
% the writing system used in this project. If there is case
% distinction be sure to mark lower case characters with \lccode
% and upper case characters with \uccode. Often case distinction
% is found in Latin-based writing systems.This list can get fairly
% fairly long. See the TeX Book for more information.
%\lccode "0061 = "0061
%\uccode "0041 = "0041


